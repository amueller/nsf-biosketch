%%%%%%%%% MASTER -- compiles the 4 sections

\documentclass[11pt,letterpaper]{article}

%%%%%%%%%%%%%%%%%%%%%%%%%%%%%%%%%%%%%%%%%%%%%%%%%%%%%%%%%%%%%%%%%%%%%%%%%
%%%%%%%%%% EXACT 1in MARGINS %%%%%%%                                   %%
\setlength{\textwidth}{6.5in}     %%                                   %%
\setlength{\oddsidemargin}{0in}   %% (It is recommended that you       %%
\setlength{\evensidemargin}{0in}  %%  not change these parameters,     %%
\setlength{\textheight}{8.5in}    %%  at the risk of having your       %%
\setlength{\topmargin}{0in}       %%  proposal dismissed on the basis  %%
\setlength{\headheight}{0in}      %%  of incorrect formatting!!!)      %%
\setlength{\headsep}{0in}         %%                                   %%
\setlength{\footskip}{.5in}       %%                                   %%
\PassOptionsToPackage{obeyspaces, spaces}{url} % allow spaces in url, allow breaking on them
\usepackage{hyperref}  % links within pdf
\hypersetup{pdftex,colorlinks=true,allcolors=blue}
\usepackage{scrextend} % labeling environment (a better description environment)
\usepackage{hypcap}    % captions for floats? not sure this does anything?
\usepackage{xspace}    % correct spacing after new commands
\usepackage{pgfgantt}  % gantt diagram (timeline)
\usepackage{booktabs}  % tables
\usepackage{sidecap}   % caption next to figure
\usepackage{titlesec}  % section header formatting (spacing)
\usepackage{enumitem}  % no space between list items
\setlist{nosep}
\usepackage{amsmath}   % math stuff
\usepackage{amssymb}   % math symbols
\usepackage[maxbibnames=99]{biblatex}
%\setcounter{biburllcpenalty}{7000} % allow breaking urls at lowercase characters
%%%%%%%%%%%%%%%%%%%%%%%%%%%%%%%%%%%%                                   %%
% define new ``required'' section heading
% which is not numbered and centered
\titleclass{\required}{straight}[\section]
\newcounter{required}
\renewcommand{\therequired}{\arabic{required}}
\titlespacing*{\required}     {0pt}{3.5ex plus 1ex minus .2ex}{2.3ex plus .2ex}
\titleformat{\required}{\filcenter\normalfont\Large\bfseries}{}{0pt}{}

% list all names in the bibliography
\renewcommand{\refname}{\hfil References Cited\hfil}                   %%
\bibliography{references}                                          %%
%%%%%%%%%%%%%%%%%%%%%%%%%%%%%%%%%%%%%%%%%%%%%%%%%%%%%%%%%%%%%%%%%%%%%%%%%

%PUT YOUR MACROS HERE
\newcommand{\sklearn}[0]{\texttt{scikit-learn}\xspace}                    %%
\preto\fullcite{\AtNextCite{\defcounter{maxnames}{200}}}

% these are the defaults for orientation
%\titlespacing*{\section}      {0pt}{3.5ex plus 1ex minus .2ex}{2.3ex plus .2ex}
%\titlespacing*{\subsection}   {0pt}{3.25ex plus 1ex minus .2ex}{1.5ex plus .2ex}
%\titlespacing*{\subsubsection}{0pt}{3.25ex plus 1ex minus .2ex}{1.5ex plus .2ex}
%\titlespacing*{\paragraph}    {0pt}{3.25ex plus 1ex minus .2ex}{1em}
% these are what we'll actually use
\titlespacing*{\section}      {0pt}{3.0ex plus 2ex minus .3ex}{2.0ex plus .2ex minus .2ex}
%\titlespacing*{\section}      {0pt}{3.5ex plus 1ex minus .2ex}{2.3ex plus .2ex}
\titlespacing*{\subsection}   {0pt}{3.0ex plus 2ex minus .3ex}{1.0ex plus .2ex minus .2ex}
%\titlespacing*{\subsubsection}{0pt}{1.0ex plus 2ex minus .3ex}{0.3ex plus .2ex minus .2ex}
\titlespacing*{\paragraph}    {0pt}{2.0ex plus 2ex minus .3ex}{1em}


\begin{document}

\title{Reply to comments on S12-SSE: Improving Scikit-learn Usability and Automation}

\section{Change of proposed work}
Due to the reduced budget, and given the current state of scikit-learn development,
we will drop the tasks 6 (Searching pipeline steps), 7 (Transformation conditions)
and 9 (Feature encodings in \sklearn) from the proposal.
A revised project timeline can be found in Figure~\ref{timeline}.
Task 6 is already implemented in the current version of scikit-learn, while
tasks 7 and 9 are in the process of being implemented, partially as part of
a different grant.

\begin{figure}
    \begin{ganttchart}[
    hgrid,
    x unit=0.26cm,
    y unit chart=.5cm,
    compress calendar,
    time slot format=isodate-yearmonth,
    bar/.append style={fill=blue!50},
    include title in canvas=false,
    bar top shift=0.2,
    bar height=.6,
    bar label node/.append style={align=left, text width=7cm},
    y unit title=.3cm
    ]{2017-10}{2020-09}
    \gantttitlecalendar[title/.style={draw=none, fill=none}]{year}\\
    \gantttitlecalendar[title/.append style={fill=black!10}, title label font=\tiny\color{black!10}]{year}\\
    \ganttbar{1 Analyze parameter ranges}{2017-10}{2018-04} \\
    \ganttbar{2 Provide default ranges}{2018-05}{2019-4} \\
    \ganttbar{3 GP-base BO}{2017-10}{2018-10} \\
    \ganttbar{4 RF-based BO}{2018-05}{2018-10} \\
    \ganttbar{5 Integrate with \texttt{auto-sklearn} BO}{2018-11}{2019-10} \\
    \ganttbar{8 Integrate conditions in \sklearn{}}{2017-10}{2018-04} \\
    \ganttbar{10 \sklearn{} $\leftrightarrow$ \texttt{auto-sklearn} sync}{2018-11}{2019-10} \\
    \ganttbar{11 Analyze meta-learning}{2018-05}{2019-04} \\
    \ganttbar{12 Meta-learning packaging}{2019-11}{2020-9}
    \end{ganttchart}
    \vspace{-5mm}
    \caption{Project timeline}%
\label{timeline}
\end{figure}

\section{Overlap with Other Funded Work}
Another project of mine relating to \sklearn has recently been funded under the title
"Extensions and Maintenance of Scikit-learn", which focusses on integration of \sklearn
with the pandas library, better support for missing and categorical data,
and better tools for understanding and visualizing models. That work does not include
any aspects of automation, benchmarking or meta-learning, and removing task 9 from
this proposal eliminates any overlap.

\section{Related Education}
There are two seperate aspects of this proposal that relate to education. The first is outreach
to contribute to open source projects, through coding sprints and collaboration
with the Women in Machine Learning and Data Science group.
The other aspect is the training of students in the use of machine learning software,
in particular the software developed as part of this project.
I am teaching an annual course on Applied Machine
Learning~\footnote{\url{https://amueller.github.io/applied_ml_spring_2017/lectures.html}}
at the Columbia Data Science Institute. This elective is part of the graduate
curriculum of the Data Science program, but is open to students from other programs.
Notably, this years course had participants from programs in Astronomy, Statistics,
Economics, Urban Development, Applied Mathematics and Finance.
This class is focused on teaching data analysis skills and software tools that
can be applied to real-world problems. Tools developed as part of this project
will be included in the course as they are completed. This will also provide
feedback for further refinement of the software.

\section{Licensing of Software Products}
All software produced as part of this project will be licensed using the BSD three-clause
licensed reproduced below:

\begin{verbatim}
Copyright (c) 2007–2017 The scikit-learn developers.
All rights reserved.

Redistribution and use in source and binary forms, with or without
modification, are permitted provided that the following conditions are met:

  a. Redistributions of source code must retain the above copyright notice,
     this list of conditions and the following disclaimer.
  b. Redistributions in binary form must reproduce the above copyright
     notice, this list of conditions and the following disclaimer in the
     documentation and/or other materials provided with the distribution.
  c. Neither the name of the Scikit-learn Developers  nor the names of
     its contributors may be used to endorse or promote products
     derived from this software without specific prior written
     permission. 

THIS SOFTWARE IS PROVIDED BY THE COPYRIGHT HOLDERS AND CONTRIBUTORS "AS IS"
AND ANY EXPRESS OR IMPLIED WARRANTIES, INCLUDING, BUT NOT LIMITED TO, THE
IMPLIED WARRANTIES OF MERCHANTABILITY AND FITNESS FOR A PARTICULAR PURPOSE
ARE DISCLAIMED. IN NO EVENT SHALL THE REGENTS OR CONTRIBUTORS BE LIABLE FOR
ANY DIRECT, INDIRECT, INCIDENTAL, SPECIAL, EXEMPLARY, OR CONSEQUENTIAL
DAMAGES (INCLUDING, BUT NOT LIMITED TO, PROCUREMENT OF SUBSTITUTE GOODS OR
SERVICES; LOSS OF USE, DATA, OR PROFITS; OR BUSINESS INTERRUPTION) HOWEVER
CAUSED AND ON ANY THEORY OF LIABILITY, WHETHER IN CONTRACT, STRICT
LIABILITY, OR TORT (INCLUDING NEGLIGENCE OR OTHERWISE) ARISING IN ANY WAY
OUT OF THE USE OF THIS SOFTWARE, EVEN IF ADVISED OF THE POSSIBILITY OF SUCH
DAMAGE.
\end{verbatim}


The BSD three-clause license and other variations of the BSD license
are recognized licenses by the Open Source
Initiative\footnote{\url{https://opensource.org/licenses}}.
The BSD three-clause license is the license currently used by the scikit-learn
project, which will facilitate integration with the existing package.
This is also the license used in other projects within the scientific Python
ecosystem, such as NumPy, SciPy, Pandas and others, and is a de-facto
standard in the community. The three-clause BSD
license is a permissive open source license (as opposed to a copy-left license
like the GPL) that allows unlimited use and modification of the code, both in a
scientific and commercial context. The open nature of the license encourages scientists
and companies to contribute code without fear of losing interlectual property rights,
such as patents, for their contributions, while encouraging software usage without
the possibly severe consequences of a copy-left license.

\section{Quantitative Usage Metrics}
We will use two main metrics to quantify the adoption of our software:
Usage in open source code published on GitHub, and contributors.
Thanks to the Google BigQuery interface for open source repositories on GitHub\footnote{\url{https://cloud.google.com/bigquery/public-data/github}} and
the Github code search feature it is possible to quanitatively analyze a large
amount of scientific software and experimentation code. This even allows for
fine-grained analysis of which features of the software are used.
For code that is contributed to the \sklearn package, we will use this code
analysis exclusively, as contributors to parts of the package are hard to
track\footnote{The number of editors of a file is not a good proxy for the
number of contributors.} For code that lives in a separate package, we will use
contributors and code analysis statistics.

\begin{itemize}
\item[Year 1] At least 10 open source projects or research projects using the provided features
\item[Year 2] At least 20 open source projects or research projects using the
provided features, at least 2 external contributors to the project.
\item[Year 3] At least 50 open source projects or research projects using the
provided features, at least 5 external contributors to the project.
\end{itemize}
\vspace{10pt}
We do not include citations into our metrics, as citations for software are
unfortunately rare, and a paper to describe the meta-learning project
could only be published at the end of the third year. Counting
citations to scikit-learn related publications are unlikely to reflect
the particular outcomes of this project.

\section{User and Community Engagement}


\end{document}
