\required{Data Management Plan}
\section{Roles and responsibilities}

\textbf{Principal Investigator: Dr.\ Andreas M\"uller}\\
PI M\"uller will oversee the data management, and ensure digital preservation of
all produced data. Dr.\ M\"uller will also ensure that all resulting materials
are deposited into a repository that performs digital preservation.
He will also ensure that all developed code is continuously version-controlled to
preserve development history.
\\\\
\textbf{Research Engineer:}\\
The Research Engineer will use the GitHub version control platform to
continuously and incrementally keep track of software development. While some
part of the created software will immediately be integrated in to the
\sklearn{} project, other parts of the software will exist in a separate
GitHub repository. The Research Engineer will also document all software and
experimental protocols within the GitHub projects using the \texttt{sphinx}
documentation generator, the standard tool in the Python development community.
The Research Engineer will implement automatic compilation of the documentation
into a website that will be publicly available for all materials outside the \sklearn{}
project. Materials contributed to the \sklearn{} project are automatically
published on the \sklearn{} website via the \sklearn{} infrastructure.

The research engineer will retrieve public machine learning data sets via the
OpenML Python API, perform the experiments, and upload the experimental results
to the OpenML platform. Large scale machine learning experiments will be
performed using the NYU High
Performance Computing clusters. The research engineer will also upload the
collective results of all experiments periodically to the figshare platform.

\section{Types of data}

The main output of this project will be Python code, which will be recorded and
shared via GitHub. Documentation of experimental procedures and documentation of the
source code will be contained in the same GitHub repository as the code.

The project will make use of the public machine learning datasets hosted on the
OpenML platform. We will use several hundred of the datasets hosted there, which will
be downloaded via the Python API of OpenML\@. The datasets are stored as ARFF
files, with the metadata stored as JSON\@. The Python API represents these
together as Python objects.

This data will be processed using a variety of algorithms from the \sklearn{}
machine learning library, as well as methods that we will implement as part of
the project. The result of the processing will be predictions made by machine
learning algorithms, their accuracy and other evaluation metrics. The
metadata of these results consists of the original dataset, machine learning
task, machine learning model that was used, and the parameters of the model.

These results and the metadata will be captured and uploaded by the OpenML
Python API to the OpenML platform, which will store it as JSON\@. We will also
separately store the collected results across many datasets and algorithms on
figshare as JSON file. This will provide a simple access mechanism as
alternative to the OpenML interface for obtaining the results. OpenML has
programming interfaces in Java, R, Python, C\# and other languages that will
allow processing of the results.


\section{Policies for access and sharing}

Source code for all software will be made available as an ongoing process
during development. All code and accompanying documentation will be licensed
under a BSD license.

Results will be pushed to OpenML immediately as part of the evaluation,
whenever possible. The results will be archived and published to figshare after
initial analysis confirmed them to be correct.
All data will be made available under the CC-0 license.

\section{Data storage and preservation of access}

The code will be part of an open source project and thereby handed to the open
source community. The initial archive will be GitHub, but the community is
expected to curate and archive the project beyond the lifetime of GitHub.
The results will be stored in the OpenML platform and on figshare, both of
which have long-term preservation guarantees.

