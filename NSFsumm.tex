\vspace{-3mm}
\required{Project Summary}
\vspace{-3mm}
The open-source machine learning library \sklearn{} has become a cornerstone of
applied machine learning and data science in academic and industrial research.
The PI M\"uller has been involved in the \sklearn{} project as co-maintainer
and core contributor for over 5 years. This proposal, if funded, will improve
ease of use of the \sklearn{} project, reduce the barrier to entry for
no-experts to use the package, and add automation features that will allow more
effective use by researchers trained in machine learning.

While the \sklearn{} project has received wide recognition for its ease of
use and extensive documentation, many areas for improvement remain.
\sklearn{} was developed for researchers in the machine learning domain,
but was later adopted by researchers across many disciplines.
Two of the main barriers to entry for domain scientists wanting to apply
machine learning are finding a suitable representation of the data to apply
machine learning, and choosing an algorithm and parameters for a particular
dataset and task.

A large amount of research has recently been performed into automatic model
selection, and systematic evaluation of machine learning system.
However, advances in these areas have not made the transition from computer
science research to being applied by domain scientists to solve practical
problems.
By providing guidance on model selection and data preprocessing via
systematic evaluation, and integrating robust implementations of automatic
algorithm selection in the established \sklearn{} ecosystem, this proposal
will allow a more effective use of machine learning across research domains.
%
\vspace{-3mm}
\required{Intellectual Merit}
\vspace{-3mm}
This proposal advances knowledge in two ways:
\begin{enumerate}
    \item By further lowering the barrier of entry for applying machine
        learning, and improving the existing tools provided by \sklearn{}, the
        proposed project will enable more researchers to adopt machine-learning
        solutions to data-driven problems.
    \item By conducting a large-scale experimental survey of existing methods,
        the project will provide guidance for machine learning practitioners
        and pointers to future directions for machine learning researchers.
\end{enumerate}
%
\vspace{-6mm}
\required{Broader Impacts}
\vspace{-3mm}
Machine learning has become a core part of many data driven research projects,
and is often implemented via open source software.
In particular the Python ecosystem of data science tools has been widely
adopted in the scientific community, both for research and in teaching.
With \sklearn{} being the primary resource for machine learning inside
the Python ecosystem, improvements to \sklearn{} will benefit all researcher
and teachers using this set of tools.
With the enhancements described in this proposal, even more people
will be able to easily apply machine learning to their problems,
without requiring extensive training in machine learning.

As a major scientific open source software package with a wide contributor
and user base, \sklearn{} is in a great position to change
the composition and values of the scientific open source ecosystem.
This proposal includes plans to host ``coding sprints'' to enlarge the number
of contributors even further, with a focus on attracting women to contributing
to open source. These events will be held in collaboration with local and
national organizations to promote women in programming and science.
