\vspace{-2mm}
\required{Project Summary}
\vspace{-2mm}
The open-source machine learning library scikit-learn has become a cornerstone of
applied machine learning and data science in academic and industrial research.
The PI M\"uller has been involved in the scikit-learn project as co-maintainer
and core contributor for over 5 years. This project will improve ease of use of the
scikit-learn project, reduce the barrier to entry for no-experts to use the package,
and add automation features that will allow more effective use by experts.

While the scikit-learn project has received wide recognition for its ease of
use and extensive documentation, many areas for improvement remain.
Scikit-learn was developed for trained machine learning researchers,
but was later adopted by researchers across many disciplines.
We identified two main barriers to entry for domain scientists wanting to apply
machine learning:
The representation of the data needed to apply machine learning, and the choice
of algorithm for a particular dataset and task.

A large amount of research has recently been performed into automatic model
selection, and systematic evaluation of machine learning system.
However, advanced in these areas have not made the transition from computer
science research to being applied by domain scientists to solve practical
problems.
By providing guidance on model selection and data preprocessing via
systematic evaluation, and integrating robust implementations of automatic
algorithm selection in the established scikit-learn ecosystem, this proposal
will allow a more effective  across research domains.
%
\vspace{-2mm}
\required{Intellectual Merit}
\vspace{-2mm}
This proposal advances knowledge in two ways:
\begin{enumerate}
    \item By lowering the barrier of entry for applying machine learning even further,
    and improving the existing tools provided by scikit-learn, will enable more
    researchers to adopt machine-learning solutions to data-driven problems.
    \item By conducting a large-scale experimental survey of existing methods,
    the project will provide guidance for machine learning practitioners
    and pointers to future directions for machine learning researchers.
\end{enumerate}
%
\vspace{-6mm}
\required{Broader Impacts}
\vspace{-2mm}
Machine learning has become a core part of many data driven research projects,
and is often implemented via open source tools.
In particular the python ecosystem of data science tools has been widely
adopted in the scientific community, both for research and in teaching.
With scikit-learn being the primary resource for machine learning inside
the python ecosystem, improvements to scikit-learn will benefit all researcher
and teachers using this set of tools.
With the enhancements described in this proposal, even more people
will be able to easily apply machine learning to their problems,
without requiring large amounts of machine learning training.

As a major scientific open source software package with a wide contributor
and user base, scikit-learn is in a great position to change
the composition and values of the scientific open source ecosystem.
This proposal includes plans to host ``coding sprints'' to enlarge the number
of contributors even further, with a focus on attracting women to contributing
to open source. These events will be held in collaboration with local an
national organizations to promote women in programming and science.
